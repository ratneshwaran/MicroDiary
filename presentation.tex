\documentclass[aspectratio=169]{beamer}
\usepackage[T1]{fontenc}
\usepackage[utf8]{inputenc}
\usepackage{lmodern}
\usepackage{listings}
\usepackage{xcolor}
\usepackage{hyperref}
\usepackage{fontawesome5}

% ── Colour palette ──────────────────────────────────────────────────────────
\definecolor{primary}{HTML}{2563EB}
\definecolor{primarydark}{HTML}{1D4ED8}
\definecolor{lightblue}{HTML}{EFF6FF}
\definecolor{muted}{HTML}{6B7280}
\definecolor{codebg}{HTML}{F8FAFC}
\definecolor{codekey}{HTML}{7C3AED}
\definecolor{codestr}{HTML}{059669}
\definecolor{codecomment}{HTML}{9CA3AF}
\definecolor{danger}{HTML}{DC2626}

% ── Theme ────────────────────────────────────────────────────────────────────
\usetheme{default}
\usecolortheme{default}
\setbeamercolor{frametitle}{fg=white, bg=primary}
\setbeamercolor{title}{fg=white}
\setbeamercolor{subtitle}{fg=white}
\setbeamercolor{author}{fg=white}
\setbeamercolor{date}{fg=white}
\setbeamercolor{normal text}{fg=black}
\setbeamercolor{structure}{fg=primary}
\setbeamercolor{block title}{fg=white, bg=primary}
\setbeamercolor{block body}{fg=black, bg=lightblue}
\setbeamercolor{alerted text}{fg=danger}
\setbeamerfont{frametitle}{size=\large, series=\bfseries}
\setbeamerfont{title}{size=\LARGE, series=\bfseries}
\setbeamertemplate{navigation symbols}{}
\setbeamertemplate{footline}[frame number]
\setbeamertemplate{itemize item}{\color{primary}\textbullet}
\setbeamertemplate{itemize subitem}{\color{muted}--}

% ── Title page background ─────────────────────────────────────────────────
\setbeamercolor{background canvas}{bg=white}

% ── Code listings ─────────────────────────────────────────────────────────
\lstdefinelanguage{TypeScript}{
  keywords={const, let, var, function, return, if, else, for, export, import,
            type, interface, extends, implements, async, await, from, of,
            true, false, undefined, null},
  keywordstyle=\color{codekey}\bfseries,
  stringstyle=\color{codestr},
  commentstyle=\color{codecomment}\itshape,
  morestring=[b]",
  morestring=[b]',
  morestring=[b]`,
  morecomment=[l]{//},
  sensitive=true
}

\lstset{
  language=TypeScript,
  backgroundcolor=\color{codebg},
  basicstyle=\ttfamily\scriptsize,
  breaklines=true,
  frame=single,
  framerule=0pt,
  rulecolor=\color{codebg},
  xleftmargin=8pt,
  xrightmargin=8pt,
  aboveskip=6pt,
  belowskip=6pt,
  showstringspaces=false,
  tabsize=2,
}

% ────────────────────────────────────────────────────────────────────────────

\title{MicroDiary}
\subtitle{An offline-first, accessible time-use diary prototype}
\author{Ratneshwaran Maheswaran}
\date{Research Assistant (Software) Interview --- UCL}

% ════════════════════════════════════════════════════════════════════════════
\begin{document}
% ════════════════════════════════════════════════════════════════════════════

% ── Title slide ─────────────────────────────────────────────────────────────
{
\setbeamercolor{background canvas}{bg=primary}
\setbeamercolor{normal text}{fg=white}
\begin{frame}[plain]
  \vspace{1.5cm}
  \begin{center}
    {\Huge\bfseries\color{white} MicroDiary}\\[0.4cm]
    {\large\color{white!80!primary} An offline-first, accessible time-use diary prototype}\\[1cm]
    {\normalsize\color{white} Ratneshwaran Maheswaran}\\[0.2cm]
    {\small\color{white!70!primary} Research Assistant (Software) Interview $\cdot$ UCL}\\[1cm]
    {\small\color{white!80!primary}
      \faGlobe\ \href{https://micro-diary.vercel.app}{micro-diary.vercel.app}
      \quad
      \faGithub\ \href{https://github.com/ratneshwaran/MicroDiary}{github.com/ratneshwaran/MicroDiary}
    }
  \end{center}
\end{frame}
}

% ── Slide 1: Overview ────────────────────────────────────────────────────────
\begin{frame}{What is MicroDiary?}

  \begin{columns}[T]
    \begin{column}{0.55\textwidth}
      \textbf{The problem}\\[0.3cm]
      Time-use studies ask participants to record activities throughout the day.
      Most tools assume internet access, mouse input, and produce messy data
      that needs cleaning post-hoc.\\[0.5cm]

      \textbf{What I built}\\[0.3cm]
      A browser-based diary that:
      \begin{itemize}
        \item Works \textbf{fully offline} after first load
        \item Keeps data \textbf{on the device} --- no backend
        \item Enforces constraints \textbf{at input time}
        \item Is usable by keyboard and screen reader users
      \end{itemize}
    \end{column}

    \begin{column}{0.42\textwidth}
      \begin{block}{Live demo}
        \centering
        \href{https://micro-diary.vercel.app}{\texttt{micro-diary.vercel.app}}
      \end{block}
      \vspace{0.4cm}
      \begin{block}{Source code}
        \centering
        \href{https://github.com/ratneshwaran/MicroDiary}{\texttt{github.com/ratneshwaran/MicroDiary}}
      \end{block}
      \vspace{0.4cm}
      \begin{alertblock}{Key constraint}
        No data leaves the device unless the participant explicitly exports it.
      \end{alertblock}
    \end{column}
  \end{columns}

\end{frame}

% ── Slide 2: User-centred design ─────────────────────────────────────────────
\begin{frame}{User-Centred Design}

  \textcolor{muted}{\small Two types of user to design for:}\\[0.5cm]

  \begin{columns}[T]
    \begin{column}{0.48\textwidth}
      \textbf{\color{primary} Participant}
      \begin{itemize}
        \item Simple single-page form
        \item Plain-language error messages
        \item Draft autosaved every 500\,ms\\
              \textcolor{muted}{\small (restores on accidental tab close)}
        \item Offline banner when connection drops
        \item Overlap detection blocks conflicting entries
        \item Gap warnings flag uncovered periods
      \end{itemize}
    \end{column}

    \begin{column}{0.48\textwidth}
      \textbf{\color{primary} Researcher}
      \begin{itemize}
        \item Every export includes:
          \begin{itemize}
            \item Schema version
            \item App version
            \item IANA timezone
            \item Stable client ID
            \item Export timestamp
          \end{itemize}
        \item Export to \textbf{JSON} (structured) or \textbf{CSV} (spreadsheets)
        \item Zod validates every entry\\
              \textcolor{muted}{\small before write and before export}
      \end{itemize}
    \end{column}
  \end{columns}

\end{frame}

% ── Slide 3: Accessibility ────────────────────────────────────────────────────
\begin{frame}{Accessibility}

  \begin{columns}[T]
    \begin{column}{0.55\textwidth}
      \textbf{Error handling pattern}
      \begin{enumerate}
        \item Submit fails $\rightarrow$ focus moves to \textbf{error summary}
        \item Summary has \texttt{role="alert"} and \texttt{tabindex="-1"}
        \item Screen reader announces all errors immediately
        \item Each error links back to its field
        \item Inline error linked via \texttt{aria-describedby}
      \end{enumerate}
      \vspace{0.4cm}
      \textbf{Other patterns}
      \begin{itemize}
        \item \texttt{:focus-visible} ring --- 3\,px blue, keyboard only
        \item Skip-to-content link (first focusable element)
        \item Semantic \texttt{<section aria-labelledby>} throughout
        \item Success messages via polite \texttt{aria-live} region
        \item \texttt{<label>} associations on every input
      \end{itemize}
    \end{column}

    \begin{column}{0.42\textwidth}
      \begin{block}{ARIA error pattern}
        \lstinputlisting[language=HTML,
          basicstyle=\ttfamily\tiny,
          keywords={input,span,div},
          keywordstyle=\color{primary}\bfseries]{snippets/aria-error.html}
      \end{block}
      \vspace{0.3cm}
      \textcolor{muted}{\small\itshape
        Demo: Tab through form $\rightarrow$ submit empty\\
        $\rightarrow$ focus jumps to error summary
      }
    \end{column}
  \end{columns}

\end{frame}

% ── Slide 4: Technologies ─────────────────────────────────────────────────────
\begin{frame}{Technologies}

  \renewcommand{\arraystretch}{1.5}
  \begin{tabular}{p{3.2cm} p{8cm}}
    \textbf{Tool} & \textbf{Why I chose it} \\
    \hline
    Vite + TypeScript &
      Fast build, full type safety. Strict mode catches bugs at compile time. \\

    \textbf{idb} &
      Typed wrapper over IndexedDB. Entries persist offline and survive restarts.
      Avoids the 5\,MB localStorage limit. \\

    \textbf{Zod} &
      Runtime schema validation. TypeScript types disappear at runtime ---
      Zod validates every database write and every export. \\

    \textbf{vite-plugin-pwa} &
      Generates a Workbox service worker automatically.
      App shell is pre-cached on first visit; loads offline thereafter. \\

    Vitest &
      30 unit tests on pure domain functions.
      Fast, zero-config with Vite. \\

    ESLint + Prettier &
      Consistent code style enforced in CI on every push. \\
  \end{tabular}

  \vspace{0.3cm}
  \textcolor{muted}{\small No UI framework — vanilla TypeScript + DOM.
  Keeps the bundle lean ($\sim$18\,kB gzip) and easy for contributors to follow.}

\end{frame}

% ── Slide 5: Code snippet ──────────────────────────────────────────────────────
\begin{frame}[fragile]{Code Snippet --- \texttt{src/domain/validation.ts}}

  \begin{columns}[T]
    \begin{column}{0.58\textwidth}
\begin{lstlisting}
export function validateFormFields(
  fields: Partial<FormFields>,
  dateEntries: DiaryEntry[],
  editingId?: string
): ValidationResult {
  const errors: FieldError[] = [];

  // 1. Structural validation via Zod
  const result = FormFieldsSchema.safeParse(fields);
  if (!result.success) {
    for (const issue of result.error.issues) {
      const fieldId = FIELD_ID_MAP[issue.path[0]];
      errors.push({ fieldId, message: issue.message });
    }
  }

  // 2. End time must be after start time
  if (fields.startTime && fields.endTime) {
    if (!isTimeBefore(fields.startTime, fields.endTime)) {
      errors.push({
        fieldId: FIELD_ID_MAP.endTime,
        message: "End time must be after start time.",
      });
    }
  }

  // 3. No overlapping entries on the same date
  const candidates = editingId
    ? dateEntries.filter((e) => e.id !== editingId)
    : dateEntries;
  const conflict = candidates.find((e) =>
    timesOverlap(fields, e)
  );
  if (conflict) {
    errors.push({ fieldId: FIELD_ID_MAP.startTime,
      message: `Overlaps with "${conflict.activity}"` });
  }

  return { valid: errors.length === 0, errors };
}
\end{lstlisting}
    \end{column}

    \begin{column}{0.38\textwidth}
      \textbf{What this demonstrates:}\\[0.3cm]

      \textbf{\color{primary} 1. Pure function}\\
      \textcolor{muted}{\small No DOM, no DB calls.
      Domain logic is completely separate from UI.}\\[0.4cm]

      \textbf{\color{primary} 2. Layered validation}\\
      \textcolor{muted}{\small Zod handles structure first,
      then business rules layer on top clearly.}\\[0.4cm]

      \textbf{\color{primary} 3. Testable}\\
      \textcolor{muted}{\small 9 unit tests cover edge cases:
      self-overlap on edit, back-to-back entries, multiple errors at once.}\\[0.4cm]

      \textbf{\color{primary} 4. Separation of concerns}\\
      \textcolor{muted}{\small Returns HTML element IDs.
      UI wires errors without domain layer knowing the DOM.}
    \end{column}
  \end{columns}

\end{frame}

% ── Slide 6: Why relevant ─────────────────────────────────────────────────────
\begin{frame}{Why This Is Relevant to the Project}

  \begin{columns}[T]
    \begin{column}{0.48\textwidth}
      \textbf{\color{primary} Data quality at collection time}
      \begin{itemize}
        \item Overlapping entries are \textbf{blocked} before saving
        \item Gaps trigger a \textbf{warning} (non-blocking)
        \item Categories come from a \textbf{fixed vocabulary}
        \item Schema version stored with every entry\\
              \textcolor{muted}{\small handles model changes between study waves}
      \end{itemize}
      \vspace{0.5cm}
      \textbf{\color{primary} Offline-first for field conditions}
      \begin{itemize}
        \item Works without internet after first load
        \item Saves locally when offline
        \item Participant exports when ready
      \end{itemize}
    \end{column}

    \begin{column}{0.48\textwidth}
      \textbf{\color{primary} Maintainable for open source}
      \begin{itemize}
        \item Domain, storage, UI in separate layers
        \item A contributor can change error display\\
              without touching validation logic
        \item A new export format doesn't touch the form
        \item 30 unit tests, CI on every push
      \end{itemize}
      \vspace{0.5cm}
      \textbf{\color{primary} Reusable patterns}
      \begin{itemize}
        \item Zod schema is the single source of truth
        \item Same schema validates writes, exports, and imports
        \item Categories defined once in \texttt{constants.ts}
      \end{itemize}
    \end{column}
  \end{columns}

\end{frame}

% ── Slide 7: Summary ──────────────────────────────────────────────────────────
{
\setbeamercolor{background canvas}{bg=primary}
\begin{frame}[plain]
  \vspace{0.8cm}
  \begin{center}
    {\Large\bfseries\color{white} Summary}\\[0.8cm]
  \end{center}

  \begin{columns}[T]
    \begin{column}{0.22\textwidth}
      \centering
      \textbf{\color{white} Accessible}\\[0.2cm]
      \textcolor{white!80!primary}{\small ARIA errors, keyboard navigation, focus management}
    \end{column}
    \begin{column}{0.22\textwidth}
      \centering
      \textbf{\color{white} Offline-first}\\[0.2cm]
      \textcolor{white!80!primary}{\small PWA + IndexedDB, works without internet}
    \end{column}
    \begin{column}{0.22\textwidth}
      \centering
      \textbf{\color{white} Research-grade}\\[0.2cm]
      \textcolor{white!80!primary}{\small Zod validation, provenance metadata, schema versioning}
    \end{column}
    \begin{column}{0.22\textwidth}
      \centering
      \textbf{\color{white} Maintainable}\\[0.2cm]
      \textcolor{white!80!primary}{\small Separated layers, 30 unit tests, open source CI}
    \end{column}
  \end{columns}

  \vspace{1cm}
  \begin{center}
    {\small\color{white!80!primary}
      \faGlobe\ \href{https://micro-diary.vercel.app}{\textcolor{white}{micro-diary.vercel.app}}
      \qquad
      \faGithub\ \href{https://github.com/ratneshwaran/MicroDiary}{\textcolor{white}{github.com/ratneshwaran/MicroDiary}}
    }
  \end{center}
\end{frame}
}

\end{document}
